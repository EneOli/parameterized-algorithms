%% Basierend auf einer Vorlage des SDQ
%% Siehe https://sdqweb.ipd.kit.edu/wiki/Dokumentvorlagen

\newcommand{\BigO}[1]{\ensuremath{\operatorname{\mathcal{O}}\bigl(#1\bigr)}}

\newcommand{\cloud}[1]{
    \begin{tikzpicture}
 
        \node[cloud,
            %draw =blue,
            %text=cyan,
            fill = gray!10,
            minimum width = 3cm,
            minimum height = 2cm] (c) at (0,0) {#1};
         
    \end{tikzpicture}
}

%% Beispiel-Präsentation
\documentclass[navbaroff]{sdqbeamer} 
\usepackage{tikz}
 
%% Titelbild
\titleimage{banner_2020_kit}

%% Gruppenlogo
\grouplogo{} 

%% Gruppenname und Breite (Standard: 50 mm)
\groupname{Proseminar Algorithmen für NP-schwere Probleme}
%\groupnamewidth{50mm}

% Beginn der Präsentation

\title[Parametrisierte Algorithmen II]{Parametrisierte Algorithmen II}
\subtitle{Kurzvortrag} 
\author[Oliver Enes]{Oliver Enes}

\date[08.\,05.\,2023]{08. Mai 2023}

% Literatur 
 
\usepackage[citestyle=authoryear,bibstyle=numeric,hyperref,backend=biber]{biblatex}
\addbibresource{presentation.bib}
\bibhang1em

\tikzstyle{every node}=[circle, draw, fill=black!50, inner sep=0pt, minimum width=4pt]
\usetikzlibrary{graphs,graphs.standard}
\usetikzlibrary{calc,shapes.callouts,shapes.arrows}

\tikzset{
  wavy/.style = {
     decorate,
     decoration = {snake, amplitude=1.5mm, segment length=8mm}
  },
  main node/.style={circle,fill=gray!25,draw,font=\sffamily\Large\bfseries}
}

\begin{document}
 
%Titelseite
\KITtitleframe

\begin{frame}{Ausgangssituation: $\mathcal{NP}$-schwere Probleme... überall $\mathcal{NP}$-schwere Probleme}
        \centering
        \includegraphics[width=5cm]{logos/np-everywhere.jpg}
    
    \visible<2->{
        \begin{questionblock}{Frage}
        Ist ein $\mathcal{NP}$-schweres Problem in der Praxis nicht effizient/schnell lösbar?
        \end{questionblock}
    }

    \visible<3->{
        \begin{brownblock}{Antwort}
            In der Praxis sind viele Instanzen $\mathcal{NP}$-schwerer Probleme effizient lösbar, weil die Instanzen \textbf{gutartig} sind!
        \end{brownblock}
    }
    
\end{frame}

\begin{frame}{Parametrisierte Algorithmen}

    \begin{itemize}
        \item Die Laufzeit eines Algorithmus wird zusätzlich zur Eingabegröße in einem \textbf{weiteren Parameter k} betrachtet.
        \item Mittels k soll die \enquote{Schwierigkeit der konkreten Probleminstanz} formalisiert werden.
        \item In diesem Vortrag: Fokus auf strukturelle Parametrisierung
    \end{itemize}

    \vspace{1cm}
    \centering
    \visible<2->{
        \textbf{\large{In welchem Teil einer Probleminstanz liegt die Schwierigkeit?}}
    }
\end{frame}

\begin{frame}{Problemkerne}

    \begin{blueblock}{Problem Vertex Cover}
        \textbf{Eingabe}: Graph $G = (V, E)$, $k \in \mathbb{N}$
        \\
        \vspace{0.075cm}
        \textbf{Parameter}: $k$
        \\
        \vspace{0.075cm}
        \textbf{Frage}: Existiert Menge $M \subseteq V$, sodass $|M| \leq k$ und jede Kante zu min. einem Knoten in $M$ adjazent ist?
    \end{blueblock}

    \begin{columns}
    \begin{column}{0.5\linewidth}
        \centering
        \visible<3->{
                \begin{tikzpicture}
            \tikzset{enclosed/.style={draw, circle, inner sep=0pt, minimum size=.1cm, fill=black}}

      \node[enclosed] (E) at (0.75,3.25) {};
      \node[enclosed] (H) at (3,4) {};
      \node[enclosed] (B) at (4.5,2) {};
      \node[enclosed] (M) at (3,0) {};
      \node[enclosed] (L) at (0.75,0.75) {};

      \node<1-3>[enclosed] (F) at (0.25,2) {};
      \node<4->[enclosed, color=red] (F) at (0.25,2) {};
      
      \node<1-3>[enclosed] (G) at (4.5,1) {};
      \node<4->[enclosed, color=red] (G) at (4.5,1) {};
      
      \node<1-3>[enclosed] (A) at (2.5,2) {};
      \node<4->[enclosed, color=red] (A) at (2.5,2) {};
      
      \node<1-3>[enclosed] (C) at (4.5,3.5) {};
      \node<4->[enclosed, color=red] (C) at (4.5,3.5) {};

          \draw[bend] (E) -- (H);
          \draw (H) -- (B);
          \draw (B) -- (M);
          \draw (E) -- (L);
          \draw (E) -- (B);
          \draw (E) -- (M);
          \draw (L) -- (H);
          \draw (L) -- (B);
          \draw (H) -- (M);
        \end{tikzpicture}
        }

    \centering
    \visible<3->{\textbf{hier $k = 3$}}
    \end{column}
    \begin{column}{0.45\linewidth}
      \visible<2->{
          \begin{greenblock}{Satz}
            Vertex Cover ist $\mathcal{NP}$-vollständig.
           \end{greenblock}
      }
      \visible<4->{
          \begin{itemize}
              \item Trotzdem sehen wir, dass wir isolierte Knoten ignorieren können.
              \item Die \enquote{Schwierigkeit} des Problems muss also in der hier größten Zusammenhangskomponente liegen.
          \end{itemize}
      }
    \end{column}
  \end{columns}
    
\end{frame}

\begin{frame}{Der Kern allen Übels}
    \begin{itemize}
        \item Wir haben die Probleminstanz verkleinert, indem wir \enquote{einfache} Teile der Instanz entfernt haben. 

        \item Reicht es, den \enquote{schweren} Teil der Instanz zu lösen, um die Instanz selbst zu lösen?

        \item Können wir das bei jedem (NP-schweren) Problem machen?
    \end{itemize}
\end{frame}

\begin{frame}[t]{Neue Problemklassen}

    \visible<2->{
        \begin{blueblock}{Problemklasse FPT}
            Die Menge aller Entscheidungsprobleme der Form $(x,k) \in \Sigma^{*} \times \mathbb{N}$ wobei k der Parameter und die Instanz in \BigO{\textcolor{red}{f(k)} \cdot \textcolor{blue}{|x|^c}}, $c \in \mathbb{N}$ lösbar ist.
        \end{blueblock}
    }

    \centering
    \visible<3->{
        \begin{center}
            \textbf{Ist jedes parametrisierte Problem FPT?}
            \\
            \visible<4->{(Spoiler: nein)}
        \end{center}
    }
    \visible<5->{
        \centering
        \textbf{Wie zeigen wir, dass ein Problem nicht FPT ist?}
        \\
        \\
        \visible<6->{
            Parametrisierte polynomielle Reduktion (W[1]-Hardness)
        }
        \\
        \\
        \vspace{0.3cm}
        \visible<7->{
            Exponential Time Hypothesis (ETH) und Strong Exponential Time Hypothesis (SETH) ermöglichen enge untere Schranken für die Laufzeit von Algorithmen.
        }
    }
\end{frame}

\end{document} 
