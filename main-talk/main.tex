%% Basierend auf einer Vorlage des SDQ
%% Siehe https://sdqweb.ipd.kit.edu/wiki/Dokumentvorlagen


%% complexity classes
\renewcommand{\P}{\ensuremath{\mathcal{P}}}
\newcommand{\NP}{\ensuremath{\mathcal{NP}}}
\newcommand{\XPT}{\ensuremath{\mathcal{XPT}}}
\newcommand{\FPT}{\ensuremath{\mathcal{FPT}}}
\newcommand{\N}{\ensuremath{\mathbb{N}}}

\newcommand{\BigO}[1]{\ensuremath{\operatorname{\mathcal{O}}\bigl(#1\bigr)}}


%% Beispiel-Präsentation
\documentclass[navbaroff]{sdqbeamer} 
\usepackage{tikz}

\usepackage[most]{tcolorbox}

\newtcolorbox{Qbox}{
enhanced,
sidebyside,
righthand width=25pt,
height from=38pt to \pdfpageheight,
colback=white,
valign=center,
overlay={\node[xshift=-30pt] at (frame.east) 
    {
        \includegraphics[height=25pt]{QM.png}
    };
  }
}

\newcommand\questionbox[1]{%
  \begin{Qbox}
        #1
  \end{Qbox}
}

\newtcolorbox{Abox}{
enhanced,
sidebyside,
righthand width=25pt,
height from=38pt to \pdfpageheight,
colback=white,
valign=center,
overlay={\node[xshift=-30pt] at (frame.east) 
    {
        \includegraphics[height=30pt]{AW.png}
    };
  }
}

\newcommand\answerbox[1]{%
  \begin{Abox}
        #1
  \end{Abox}
}

%% Titelbild
\titleimage{banner_2020_kit}

%% Gruppenlogo
\grouplogo{} 

%% Gruppenname und Breite (Standard: 50 mm)
\groupname{Proseminar Algorithmen für NP-schwere Probleme}
%\groupnamewidth{50mm}

% Beginn der Präsentation

\title[Parametrisierte Algorithmen II]{Parametrisierte Algorithmen II}
\subtitle{Hauptvortrag} 
\author[Oliver Enes]{Oliver Enes}

\date[05.\,06.\,2023]{05. Juni 2023}

% Literatur 
 
\usepackage[citestyle=authoryear,bibstyle=numeric,hyperref,backend=biber]{biblatex}
\usepackage{tikz}

\tikzstyle{every node}=[circle, draw, fill=black!50, inner sep=0pt, minimum width=4pt]
\usetikzlibrary{graphs,graphs.standard}
\usetikzlibrary{calc,shapes.callouts,shapes.arrows}


\addbibresource{presentation.bib}
\bibhang1em

\begin{document}
 
%Titelseite
\KITtitleframe

\begin{frame}[t]{Wofür parametrisierte Algorithmen?}
    
    \begin{itemize}
        \item Viele Probleme sind \NP-schwer
    \end{itemize}

    \visible<2->{
        \vspace{20pt}
        \questionbox{
            \centering
            \textbf{Ist damit jede Instzanz eines \NP-schweren Problems nicht effizient lösbar?}
        }
    }
    
    \visible<3->{
        \answerbox{
            \centering
            \textbf{
                In der Praxis sind viele Instanzen \NP-schwerer Probleme effizient lösbar, weil die Instanzen \textcolor{blue}{gutartig} sind!
            }
        }
    }

    \visible<4->{
        \vspace{20pt}
        \centering
        \textbf{Parametrisierte Algorithmen formalisieren Gutartigkeit}
    }

\end{frame}

\begin{frame}{Parametrisiertes Problem}

    \begin{blueblock}{Parametrisiertes Problem}
        Ein parametrisiertes Problem ist eine Sprache $ L \subseteq (\Sigma^*, \textcolor{blue}{\N}) $ wobei $\Sigma$ ein endliches Alphabet.
        \\
        Die zweite Komponente heißt \textcolor{blue}{Parameter}.
    \end{blueblock}

    
    \visible<2-3>{
        \large{\textbf{Beispiele:}}
            \begin{itemize}
                \item $k$-Vertex-Cover mit Parameter $k$
                \item Subset Sum mit Parameter Summe aller Elemente
                \item 3Color mit Parameter größte Clique
                \item k-Vertex-Cover mit Parameter $k = min_{v \in V(G)}{deg(v)}$
            \end{itemize}
    }

    \visible<3>{
        \vspace{5pt}
        \large{\textbf{Anmerkungen:}}
        \begin{itemize}
            \item Parameter i.A. nicht eindeutig / frei wählbar
        \end{itemize}
    }
\end{frame}

\begin{frame}{Parametrisierte Komplexität}
    \begin{blueblock}{Problemklasse \FPT}
        Ein parametrisiertes \textbf{Entscheidungsproblem} P heißt \FPT  (Fixed-Parameter-Tractable),
        wenn ein Algorithmus existiert, der Probleminstanzen (x, k) in \BigO{\textcolor{blue}{f(k)} \cdot \textcolor{red}{|x|^c}} löst, wobei $ c \in \N $ und $f$ beliebige berechenbare Funktion.
        
        \vspace{10pt}
        Die Menge aller solcher Probleme bildet die Problemklasse \FPT
    \end{blueblock}  
    
    \visible<2-3>{
        \vspace{5pt}
        \large{Beispiele:}
        \begin{itemize}
            \item \BigO{\textcolor{blue}{(43k^2+k)} \cdot \textcolor{red}{|x|^2}}
            \item \BigO{\textcolor{blue}{2^k} \cdot \textcolor{red}{(|x|^3+42|x|)}}
            \item \BigO{\textcolor{red}{|x|^{123}+12}}
            \item \BigO{\textcolor{blue}{\varphi(k, k - 5, k - \sqrt{k})} \cdot \textcolor{red}{\log(|x + 1|)}} wobei $\varphi$ die Ackermann-Funktion.
        \end{itemize}
    }

    \visible<3>{
        \begin{itemize}
            \item \textbf{Beobachtung:} Für festes k ist die Laufzeit polynomiell in der Eingabegröße.
        \end{itemize}
    }

\end{frame}


\begin{frame}{Ein \NP-schweres Problem: Vertex-Cover}
    \begin{blueblock}{Problem k-Vertex-Cover}
        \textbf{Eingabe}: Graph $G = (V, E)$, $k \in \mathbb{N}$
        \\
        \vspace{0.075cm}
        \textbf{Parameter}: $k$
        \\
        \vspace{0.075cm}
        \textbf{Frage}: Existiert Menge $M \subseteq V$, sodass $|M| \leq k$ und jede Kante zu min. einem Knoten in $M$ adjazent ist?
    \end{blueblock}
    
    \begin{tikzpicture}
        %\tikzset{enclosed/.style={draw, circle, inner sep=0pt, minimum size=.1cm, fill=black}}

  \node (E) at (0.75,3.25) {};
  \node (H) at (3,4) {};
  \node (B) at (4.5,2) {};
  \node (M) at (3,0) {};
  \node (L) at (0.75,0.75) {};

  \node<1-3> (F) at (0.25,2) {};
  \node<4->[color=red] (F) at (0.25,2) {};
  
  \node<1-3> (G) at (4.5,1) {};
  \node<4->[color=red] (G) at (4.5,1) {};
  
  \node<1-3> (A) at (2.5,2) {};
  \node<4->[color=red] (A) at (2.5,2) {};
  
  \node<1-3> (C) at (4.5,3.5) {};
  \node<4->[color=red] (C) at (4.5,3.5) {};

      \draw[bend] (E) -- (H);
      \draw (H) -- (B);
      \draw (B) -- (M);
      \draw (E) -- (L);
      \draw (E) -- (B);
      \draw (E) -- (M);
      \draw (L) -- (H);
      \draw (L) -- (B);
      \draw (H) -- (M);
    \end{tikzpicture}

\end{frame}

\end{document} 
